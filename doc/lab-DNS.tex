%%%%%%%%%%%%%%%%%%%%%%%%%%%%%%%%%%%%%%%%%%%%%%%%%%%%%%%%%%%%%%%%%%%%%%%%%%%
% FILE         : lab-DNS.tex
% AUTHOR       : (C) Copyright 2014 by Peter C. Chapin
% LAST REVISED : 2014-01-30
% SUBJECT      : DNS lab for Network Programming at Vermont Technical College
%
% In this lab the students will write a program that queries a DNS by constructing the queries
% themselves (and launching them as UDP packets).
%
%       Peter C. Chapin
%       Computer Information Systems Department
%       Vermont Technical College
%       Williston, VT. 05495
%       PChapin@vtc.vsc.edu
%%%%%%%%%%%%%%%%%%%%%%%%%%%%%%%%%%%%%%%%%%%%%%%%%%%%%%%%%%%%%%%%%%%%%%%%%%%

% ++++++++++++++++++++++++++++++++
% Preamble and global declarations
% ++++++++++++++++++++++++++++++++
\documentclass[twocolumn]{article}
% \pagestyle{headings}
\setlength{\parindent}{0em}
\setlength{\parskip}{1.75ex plus0.5ex minus0.5ex}

% +++++++++++++++++++
% The document itself
% +++++++++++++++++++
\begin{document}

% ----------------------
% Title page information
% ----------------------
\title{UDP and the Domain Name System}
\author{\copyright\ Copyright 2014 by Peter C. Chapin}
\date{Last Revised: January 30, 2014}
\maketitle

\section*{Introduction}

In this lab you will explore UDP and the domain name system by writing a program that does DNS
queries manually. Although there are well known library functions for sending queries to a name
server (\texttt{gethostbyname()} and \texttt{gethostbyaddr()}), you will not use those functions
in this lab. By constructing the DNS query and by interpreting the response yourself you will
gain more insight into how the domain name system works. In addition, since most DNS
transactions use UDP, this lab will give you some exposure to this important, low level
transport protocol.

\section{The Program}

Your program should take two command line parameters. The first parameter will be the IP address
of the name server you wish to query. The second parameter will be the name you wish to have
converted into an IP address. A typical invocation of the program might be

\begin{verbatim}
$ dns 155.42.16.33 lemuria.cis.vtc.edu
155.42.234.80
\end{verbatim}

In this example I am using one of the main VTC name servers to handle the query. I suggest that
you use this server in your testing as well. If you accidently send the server a malformed
packet, the server should just return an error code in a normal way. If it does otherwise (for
example if it crashes), that would be a serious security issue. You can assume the main VTC
name servers are robust against such problems.

Many names have multiple IP addresses associated with them. Your program only needs to report
the first one but it should not be confused by any additional addresses that might be present.
Also your program does not need to provide any facilities for doing reverse lookups (converting
IP addresses into names). The program \texttt{nslookup} provides all these features and more.
Our purpose is not to duplicate the functionality of \texttt{nslookup} but instead to explore
how such a program might be created.

Refer to your class notes for details on the format of a DNS query and a DNS response. There are
many RFCs pertaining to the domain name system. RFC-1034 and RFC-1035 are the primary documents
describing the system. RFC-1034 describes basic concepts and RFC-1035 describes implementation
details, including the particulars of the DNS protocol. You should review these RFCs; you may
need to consult them as you build your program.

Since preparing the DNS packets will take a resonable amount of code, you should break your
program up into several suitable functions. I suggest writing a function that constructs a
request packet, a function to send the request, a function that receives the response, and
another function that disects the response looking for the desired answer. Be sure you include
appropriate error handling on all network related calls. Try to build a robust program.

Because UDP is no more reliable than IP, you will need to deal with the possibility that your
packets might be lost forever on the network. If you call the UDP \texttt{recvfrom()} function
without setting up a time out your program might block forever waiting for a packet that never
comes. On Unix systems you may deal with this time out issue by either using the
\texttt{alarm()} function and setting up a signal handler for the SIGALRM signal, or by setting
an appropriate socket option. On Windows systems it is necessary to use \texttt{select} on the
socket handle to determine when new data arrives. The \texttt{select} function allows a time out
interval to be specified (you can also use \texttt{select} on Unix, but it is more complicated
to set up than the other solutions).

Note that you can check your query before you have written any code to process the reply. Use
the network analyzer to observe the UDP packets being emitted by your system. The network
analyzer software will recognize and decode the DNS query that your program created---assuming
you did it right---thus allowing you to verify its format. You can also observe the DNS reply
coming back from the name server. I encourage you to do this.

\section{Questions to Ponder}

The following questions are intended to be food for thought. In your report you may wish to
address these questions or similar issues.

\begin{enumerate}
  
\item How many packets are exchanged between your DNS client and the server? If you had used a
TCP connection instead (about) how many packets would you have expected to see?
  
\item Your query and the name server's response could both fit into a single datagram. What is
the significance of this?
  
\item What would be involved in modifying your program so that it could use several different
name servers? Suppose it read name server addresses out of a configuration file and tried them
in order, if necessary, until it found one that responded to the query. How much network related
setup would you need to do in order to communicate with each name server? You don't need to make
these modifications; just describe in general what you would need to change.
  
\end{enumerate}

\section{Report}

For this lab your report should include sections that briefly describe the DNS protocol in
general, describe your software, discuss how you tested your system and what you observed, and a
conclusion. Imagine that your audience is other programmers who might want to modify your
software and who are only familiar with the DNS protocol in a general way.

You can address the questions in the previous section in your report, but \emph{don't simply
  answer the questions directly}. Instead integrate your answers into your discussion smoothly.
The intent of these questions is to give you some ideas about topics to elaborate on in your
report; you should regard them as guides. You are free to adjust exactly what you say in your
report according to what seems best to you.

\end{document}

